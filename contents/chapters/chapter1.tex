\chapter{Introducción}
\hrule \bigskip \vspace*{1cm}
%------------------------------------------------------------------------


Este es el ejemplo de un capítulo de tu tesis. Aquí deberías
escribir la introducción de tu tesis. Junto con este archivo
deberás encontrar otros tutoriales con más información de cómo
utilizar los diferentes paquetes y su sintaxis.

Esta es una ``posible'' estructura de tesis recomendada. Los títulos
de los capítulos solo son referentes a lo que deben tener el
contenido, pero no es necesariamente la más apropiada a tu tesis,
los expertos recomiendan no sobrecargar mucho la tesis con
información externa, es decir el conocido ``Marco teórico'' en el
capítulo 2 ya que esta puede ser ampliamente referenciado en la
bibliografía.

\section{Contexto y Motivación}

¿cual es el ámbito en que esto se desenvuelve? y ¿que necesidad
existe para motivar una investigación en tu tema?

\section{Definición del problema}

Así como ya esta, ¿que problema existe actualmente?, respecto a lo
que quieres proponer o mostrar.

\section{Justificación}

¿porque estas desarrollando esta tesis?

\section{Objetivos}

¿Que pretendes obtener o resolver? y en los objetivos específicos
detallar cada una de las tareas que realizaras, en este punto debes
ser muy exacto y concreto(recuerda que en base a esto justificaras
si estas obteniendo resultados)

\section{Organización de la tesis}

Una breve descripción de cada uno de los capítulos que estas
desarrollando desde el CAP 2 hasta el capitulo antes del apéndice.
